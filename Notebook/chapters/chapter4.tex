\section{Tables}

% Normal Table
\subsection{Simple Table}
\begin{table}[h]
\centering
\begin{tabular}{|>{}c|c|c|}
\hline
\rowcolor{tableheader}
\textbf{Column 1} & \textbf{Column 2} & \textbf{Column 3} \\
\hline Item 1 & Item 2 & Item 3 \\
\hline Item 4 & Item 5 & Item 6 \\
\hline
\end{tabular}
\caption{A simple table}
\end{table}

\subsection{Colored Table}
\begin{table}[h]
\centering
\begin{tabular}{|>{\columncolor{tableheader}}c|c|c|}
\hline
\rowcolor{tableheader}
\textbf{Column 1} & \textbf{Column 2} & \textbf{Column 3} \\
\hline Item 1 & Item 2 & \cellcolor{codehighlight} Item 3 \\
\hline Item 4 & \cellcolor{codehighlight} Item 5 & Item 6 \\
\hline
\end{tabular}
\caption{A colored table}
\end{table}

% Headerless table
\subsection{Headless Table}
\begin{center} % This is same as /centering
\normalsize % or \footnotesize, \scriptsize, \tiny
\begin{tabular}{|c|c|c|}
\hline Item 1 & Item 2 & Item 3 \\
\hline Item 4 & Item 5 & Item 6 \\
\hline Item 7 & Item 8 & Item 9 \\
\hline
\end{tabular}
\end{center}

% Borderless table
\subsection{Borderless Table}
\begin{center} % This is same as /centering
\normalsize % or \footnotesize, \scriptsize, \tiny
\begin{tabular}{ccc}
\hline Item 1 & Item 2 & Item 3 \\
\hline Item 4 & Item 5 & Item 6 \\
\hline Item 7 & Item 8 & Item 9 \\
\hline
\end{tabular}
\end{center}

% Complete Borderless table
\subsection{Complete Borderless Table}
\begin{center} % This is same as /centering
\normalsize % or \footnotesize, \scriptsize, \tiny
\begin{tabular}{ccc}
Item 1 & Item 2 & Item 3 \\
Item 4 & Item 5 & Item 6 \\
Item 7 & Item 8 & Item 9 \\
\end{tabular}
\end{center}

% Normal Table color
\subsection{Headless Table with Alternating Row Colors}
\raggedright
\large
\begin{tabular}{|c|c|c|}
\hline
\rowcolor{codehighlight} Item 1 & Item 2 & Item 3 \\
\hline Item 4 & Item 5 & Item 6 \\
\hline
\rowcolor{codehighlight} Item 7 & Item 8 & Item 9 \\
\hline
\end{tabular}
